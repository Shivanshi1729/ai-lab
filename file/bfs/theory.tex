Breadth-first search (BFS) is an algorithm for searching a graph data
structure for a node that satisfies a given property. It starts at the
graph root and explores all nodes at the present depth prior to moving
on to the nodes at the next depth level. Extra memory, usually a queue,
is needed to keep track of the child nodes that were encountered but not
yet explored.

\subsubsection*{Algorithm}

\begin{itemize}
    \item Create an empty queue q
    \item Append the source node to q
    \item Loop while q is not empty
          \begin{itemize}
              \item tempNode <- q.deque()
              \item Enqueue tempNode's children (first left then right children) to q
          \end{itemize}
\end{itemize}

\subsubsection*{Analysis:}

\begin{itemize}
    \item $V$ - number of vertices
    \item $E$ - number of edges
    \item \textbf{Time Complexity:} $O(V+E)$
    \item Complete search
\end{itemize}